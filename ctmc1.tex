\documentclass{article}
\usepackage{amsmath}
\usepackage{graphicx} % Required for inserting images
\usepackage{hyperref}
\title{Exercise 1}
\author{Haim Lavi, 038712105}
\date{May 2025}

\begin{document}

\maketitle

a. We write the discrete time transition matrix:
\[
P=(p_{ij})=\begin{pmatrix}
    0 & 1 & 0 \\
    0.5 & 0 & 0.5 \\
    0 & 1 & 0 \\
\end{pmatrix}
\]
then $q_{ij}=p_{ij}\cdot\lambda_i$ and $q_{ii}=-\lambda_i$, hence
\[
Q=(q_{ij})=\begin{pmatrix}
    -1 & 1 & 0 \\
    1 & -2 & 1 \\
    0 & 3 & -3 \\
\end{pmatrix}
\]
b. We find the only solution to the equation $\pi{Q}=0$, where $\pi=\begin{pmatrix}
    \pi_1 & \pi_2 & \pi_3
\end{pmatrix}$, such that $\pi_1+\pi_2+\pi_3=1$. This gives the following system of equations:
\begin{flalign}
-\pi_1+\pi_2=0\\
\pi_1-2\pi_2+3\pi_3=0\\
\pi_2-3\pi_3=0
\end{flalign}
Hence, $\pi=\begin{pmatrix}
    3\pi_3 & 3\pi_3 & \pi_3
\end{pmatrix}=\begin{pmatrix}
    \frac{3}{7} & \frac{3}{7} & \frac{1}{7}
\end{pmatrix}$

We check by solving the equation
\[\tilde\pi{P}=\tilde\pi\]
where $\tilde\pi_1+\tilde\pi_2+\tilde\pi_3=1$, giving the following system:
\begin{flalign}
\frac{\tilde\pi_2}{2}=\tilde\pi_1\\
\tilde\pi_1+\tilde\pi_3=\tilde\pi_2\\
\frac{\tilde\pi_2}{2}=\tilde\pi_3
\end{flalign}
Hence, $\tilde\pi=\begin{pmatrix}
    \tilde\pi_1 & 2\tilde\pi_1 & \tilde\pi_1
\end{pmatrix}=\begin{pmatrix}
    \frac{1}{4} & \frac{1}{2} & \frac{1}{4}
\end{pmatrix}$.

$\begin{pmatrix}
    \pi_1\cdot\lambda_1 & \pi_2\cdot\lambda_2 & \pi_3\cdot\lambda_3
\end{pmatrix}=\begin{pmatrix}
    \frac{3}{7}\cdot{1} & \frac{3}{7}\cdot{2} & \frac{1}{7}\cdot{3}
\end{pmatrix}=\begin{pmatrix}
    \frac{3}{7} & \frac{6}{7} & \frac{3}{7}
\end{pmatrix}=\frac{12}{7}\tilde\pi$, so the two stationary vectors are identical up to a scalar multiple.

c. Code is in the following git repository: \href{https://github.com/HaimL76/ctmc1.git}{ctmc1}.

We present the calculations for the states in three methods:
\begin{enumerate}
    \item Using the Gillespie simulation (labeled \textbf{empirical}).
    \item Using the equation $\pi{Q}=0$, as done in b (labeled \textbf{stationary}).
    \item Calculating $P(t)=e^{tQ}$ (labeled as \textbf{Pt}).
\end{enumerate}

There are several possible sources for an error:
\begin{enumerate}
    \item 
\end{enumerate}
\includegraphics{ctmc1.png}
\end{document}
