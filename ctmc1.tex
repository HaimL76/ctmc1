\documentclass{article}
\usepackage{amsmath}
\usepackage{graphicx} % Required for inserting images
\usepackage{hyperref}
\title{Exercise 1}
\author{Haim Lavi, 038712105}
\date{May 2025}

\begin{document}

\maketitle

a. We write the discrete time transition matrix:
\[
P=(p_{ij})=\begin{pmatrix}
    0 & 1 & 0 \\
    0.5 & 0 & 0.5 \\
    0 & 1 & 0 \\
\end{pmatrix}
\]
then $q_{ij}=p_{ij}\cdot\lambda_i$ and $q_{ii}=-\lambda_i$, hence
\[
Q=(q_{ij})=\begin{pmatrix}
    -1 & 1 & 0 \\
    1 & -2 & 1 \\
    0 & 3 & -3 \\
\end{pmatrix}
\]
b. We find the only solution to the equation $\pi{Q}=0$, where $\pi=\begin{pmatrix}
    \pi_1 & \pi_2 & \pi_3
\end{pmatrix}$, such that $\pi_1+\pi_2+\pi_3=1$. This gives the following system of equations:
\begin{flalign}
-\pi_1+\pi_2=0\\
\pi_1-2\pi_2+3\pi_3=0\\
\pi_2-3\pi_3=0
\end{flalign}
Hence, $\pi=\begin{pmatrix}
    3\pi_3 & 3\pi_3 & \pi_3
\end{pmatrix}=\begin{pmatrix}
    \frac{3}{7} & \frac{3}{7} & \frac{1}{7}
\end{pmatrix}$

We check by solving the equation
\[\tilde\pi{P}=\tilde\pi\]
where $\tilde\pi_1+\tilde\pi_2+\tilde\pi_3=1$, giving the following system:
\begin{flalign}
\frac{\tilde\pi_2}{2}=\tilde\pi_1\\
\tilde\pi_1+\tilde\pi_3=\tilde\pi_2\\
\frac{\tilde\pi_2}{2}=\tilde\pi_3
\end{flalign}
Hence, $\tilde\pi=\begin{pmatrix}
    \tilde\pi_1 & 2\tilde\pi_1 & \tilde\pi_1
\end{pmatrix}=\begin{pmatrix}
    \frac{1}{4} & \frac{1}{2} & \frac{1}{4}
\end{pmatrix}$.

$\begin{pmatrix}
    \pi_1\cdot\lambda_1 & \pi_2\cdot\lambda_2 & \pi_3\cdot\lambda_3
\end{pmatrix}=\begin{pmatrix}
    \frac{3}{7}\cdot{1} & \frac{3}{7}\cdot{2} & \frac{1}{7}\cdot{3}
\end{pmatrix}=\begin{pmatrix}
    \frac{3}{7} & \frac{6}{7} & \frac{3}{7}
\end{pmatrix}=\frac{12}{7}\tilde\pi$, so the two stationary vectors are identical up to a scalar multiple.

c. Code is in the following git repository: \href{https://github.com/HaimL76/ctmc1.git}{ctmc1}.

We present the calculations for the states in three methods:
\begin{enumerate}
    \item Using the Gillespie simulation (labeled \textbf{empirical}).
    \item Using the equation $\pi{Q}=0$, as done in b (labeled \textbf{stationary}).
    \item Calculating $P(t)=e^{tQ}$ (labeled as \textbf{Pt}).
\end{enumerate}

We observe that $e^{tQ}$ converges to the stationary distribution only after a few steps, with a negligible error. 

On the other hand, the calculation of the stationary distribution by the Gillespie simulation is not as obvious, and depends on our definition of the allowed error for convergence, and the way that we measure this error. For this, I enable in my code a parameter for the error, and a parameter for counting consecutive successes, meaning consecutive samples for which the error is less than the allowed error. The $t$ from which this count begins will be considered as an optimal $t$, and running this simulation over and over again, and building a histogram of the number of simulations for the different optimal $t$ values, we observe that for some values of error and count of successes, we obtain a histogram that behaves very generally as a normal distribution.

\includegraphics{no-pt_ctmc1.png}
\includegraphics{no-pt_error_ctmc1.png}
\includegraphics{no-pt_hist_ctmc1.png}
\newpage
The following figures show also the calculation of $e^{tQ}$ and the log of error.
\includegraphics{with-pt_ctmc1.png}
\includegraphics{with-pt_error_ctmc1.png}
\end{document}
